\documentclass[a4paper, twocolumn, 11pt, twoside]{article}
\usepackage{url}
\usepackage{mathtools}
\date{\small{18, Března, 2023}}
\title{\Huge{\textbf{Fractions64}}}
\author{Ondrej Nedojedly}

\begin{document}
\maketitle
\section{Úvod}
Tato dokumentace obsahuje podrobný přehled knihovny 
Fractions64, což je knihovna jazyka C (používajícího standard 99) pro 
pro práci se zlomky s 64bitovou přesností celých čísel. 
Knihovna obsahuje funkce pro vytváření, 
manipulaci a provádění operací se zlomky.

Knihovna je navržena tak, aby se snadno používala a integrovala
do projektů v jazyce C a poskytuje celou řadu funkcí.
umožňující vývojářům pracovat se zlomky v prostředí
flexibilním a efektivním způsobem. Kromě toho knihovna
obsahuje funkce pro zjednodušení zlomků, převody zlomků a další funkce.
zlomků na čísla s pohyblivou řádovou čárkou a další.

Tato dokumentace poskytuje přehled o knihovně
včetně příkladů použití jednotlivých funkcí.
a popisuje datové struktury knihovny.
a rozhraní API. Je určena vývojářům, kteří hledají
pracovat se zlomky ve svých projektech v jazyce C a kteří chtějí pracovat se zlomky.
porozumět tomu, jak používat knihovnu Fractions64, aby 
k dosažení tohoto cíle.

\section{Stažení}
Tento projekt k dispozici ke stažení na adrese
mého githubu\footnote{https://www.github.com/ondranedo/fractions}.
Projekt můžete naklonovat přímo do svého vývojového prostředí IDE\footnote{Integrované vývojové prostředí}, nebo můžete
naklonovat projekt pomocí příkazu git clone. Pokud však chcete jen
použít knihovnu, můžete si složku se zdrojovými kódy stáhnout ručně, 
jsou tam koneckonců jen dva zdrojové soubory.

\begin{list}{}
\item \small{\emph{fractions64.h}}
\item \small{\emph{fractions64.c}}
\end{list}
\vspace{13pt}

\section{Jak nastavit projekt}
Pokud jste si zvolisi jinou variant sestavení, dovloste mi podotknou že
že je třeba projekt sestavit jako .lib. Celé toto muže udělat nástroj CMake\footnote{https://cs.wikipedia.org/wiki/CMake}, který vytvoří .lib 
soubor za vás, ale klidně použijte vlastní nástroje pro sestavení.

Pro ty, kteří si zvolili sestavovací nástroj CMake, stačí přímo
naklonovat tuto knihovnu do navržené složky pro externí nástroje.
knihovny (něco jako \textit{/vendor}). A do své 
root CMakeLists.txt přidejte tento příkaz: \textit{addSubdirectory([your\_specified\_path])}.
Musíte také zahrnout přidruženou hlavičku, která se nachází ve složce
\textit{/include}. A nakonec
musíte do svého spustitelného souboru zahrnout soubor .lib. název souboru
lib je \textit{frac}.

Pro ty, kteří používají jiný způsob sestavování, mějte na paměti, že
vaše verze překladače a OS\footnote{Operační systém} nemusí být podporovány, překladače 
na kterých byl testován, jejich přidružená verze a operační systém jsou uvedeny v seznamu
zde:
\begin{itemize}
\item \textbf{Windows}\\
Nebylo testováno.

\item \textbf{MacOS}\\
Úspěšně otestováno a vyvinuto na Clangu\footnote{https://cs.wikipedia.org/wiki/Clang}
verze \textit{Apple clang verze 14.0.0 (clang-1400.0.29.202)}
pro taget \textit{arm64-apple-darwin22.2.0} pomocí verze CMake:
\textit{3.25.1}: \textit{3.25.1}.

\item \textbf{Linux} \\
Nebylo testováno.

\end{itemize}

Existuje však alternativa pro ty, kteří se nesnaží
knihovnu vyvíjet. Pro tyti lidi je zde již předvytvořen
projekt pro různá IDE. Tyto projekty můžele nalézt ve 
složce \textit{/build}

\begin{enumerate}
    \item{\textbf{RedPanda}}
    \item{\textbf{Visual Studio 2023}}
\end{enumerate}

V těchto projektech najdete i předvytvořené demo soubory,
pro ujasnění funkcí knihovny. Tyto spustitelné soubory je
třeba ale předem zkompilovat.

\section{API}
API této knihovny je dodržováno striktním dodržováním
pojmenovávací konvencí. Ta je následující:
\begin{enumerate}
    \item Všechna funkce a datové typy začínají jménem
    \textbf{fracXxx}.
    \item Vrací li funkce pamět vytvořenou na haldě\footnote{https://cs.wikipedia.org/wiki/Halda}
    má ve svém názvu nakonci \textbf{H}. Př: \textit{fracAddFraction64H}
    \item Za název \textbf{frac} je vždy operace kterou chceme vykovávat.
\end{enumerate}

\subsection{Vytváření zlomku}
Zlomky je možno vytvořit dvěma způsoby, a to:
\begin{enumerate}
    \item Skrze funkce \textbf{Create}
    \item Skrze funkce \textbf{GetCpy}
\end{enumerate}
Metoda \textbf{Create} se dá rozdělit na dvě varianty,
a to na vytváření pomocí jmenovatele s čitatele, a nebo
pomocí desetinného čísla pomocí funkce \textbf{CreateFloat}.

Jak metoda \textbf{GetCpy} tak metoda \textbf{Create} podporují \textbf{H} variantu, která vrátí adresu 
zlomku, který byl vytvořen na haldě

\subsection{Ničení zlomku}
Pokud zlomek již nebude potřebován je jej třeba zničit 
skrye funkce \textbf{Destroy}. A jeli zlomek uchováván jako adresa na haldu,
je třeba zavolat funcki \textbf{Destroy} s \textbf{H} variantou.

\subsection{Getters a Setters}
Nastavují hodnotu zlomku, či získají hodnotu ze zlomku.

\begin{itemize}
    \item \textbf{fracGetA} vrací čitatel zlomku.
    \item \textbf{fracGetB} vrací jmenovatel zlomku.
    \item \textbf{fracSet} nastavi jmenovatel a čitatel zlomku.
    \item \textbf{fracSetA} nataví čitatel zlomku.
    \item \textbf{fracSetB} nataví jmenovatel zlomku.
\end{itemize}

\subsection{Matematické operace}

Každá funkce má \textbf{H} variantu, která vrátí adresu 
zlomku, který byl vytvořen na haldě, a variantu \textbf{Overwrite},
jež přepíše první zlomek, který byl dán jako parametr pro volání 
funkce s novými hodnotami.
\begin{itemize}
    \item \textbf{fracAdd} sečte zlomky.
    \item \textbf{fracSub} odečte zlomky.
    \item \textbf{fracMul} vynásobí zlomky.
    \item \textbf{fracDiv} vydělí zlomky.
    \item \textbf{fracPow} dá zlomek na mocninu jiného zlomku.
\end{itemize}
\subsection{Záměna jmenovatele a čitatele}
Chceme li zamenit jmenovatele a čitatele zlomku, udělat takzvanou obrácenou hodnotu
voláme funkci \textbf{fracSwitchAB}, chceme li zaměnit čitatele či jmenovatele u dvou zlomků, voláme 
funkce \textbf{fracSwitchA} pro čitatele a \textbf{fracSwitchB} pro jmenovatele.

\subsection{Výpis zlomku do souboru}
Chceme li vypsat zlomek do souboru konzole či do námi otevřeného souboru,
můžeme využít funkce \textbf{Dump} jejiž první parametr je onen soubor.
Druhý parametr je adresa zlomku, a třetím parametrem je formát. Je-li formát nastaven na hodnotu \textbf{NULL} je využit defaultni formát: $a/b$.
Chceme li nastavit náš vlastní formát použijeme pro čitatele \texttt{\textbf{\%a}} a jmenovatele \texttt{\textbf{\%b}} př.:

$$\frac{5}{12} \xrightarrow{\texttt{"[\%a, \%b] \%b ano"}} \texttt{[5, 12] 12 ano}$$

\subsection{Další funkce}
\begin{itemize}
    \item \textbf{fracCom} porovná dva zlomky. Vrací $0$ jsou li stejné, $1$ je-li první zlomek větší, a $-1$ je-li druhý zlomek větší.
    \item \textbf{Simplify} převede zlomek na jeho zjednoduší variantu.
    \item \textbf{Float} vráti zlomek v desetinném čísle.
    \item \textbf{LCD} vrátí nejmenší společný jmenovatel pro dva zlomky.
    \item \textbf{SetLogLvl} nastaví úroveň výpisu.
    Defaultní úroveň je nastavena na \textbf{FRAC\_ERROR}.
    \begin{enumerate} 
        \item \textbf{FRAC\_ALL} vypíše všechny zprávy.
        \item \textbf{FRAC\_WARNING} vypíše všechny varování a chybové stavy.
        \item \textbf{FRAC\_ERROR} vypíše pouze chybové stavy.
        \item \textbf{FRAC\_NONE} nevypíše nic.
    \end{enumerate}
    \item \textbf{SetDefaultSimplification} defaultní stav automatického zjednodušováni zlomků je \textbf{FRAC\_TRUE}, možno vypnout defaultní zjednodušováni zlomků posláním parametru \textbf{FRAC\_FALSE}.
    \item \textbf{PowDouble} vypočte desetinnou hodnotu na mocninu zlomku.

\end{itemize}

\end{document}